%!TEX root = ../main.tex

%%%%%%%%%%%%%%%%%%%%%%%%%%%%%%%%%%%%%%%%%%%
%%%%%%%%%% ГОСТОВСКИЕ ПРИБАМБАСЫ %%%%%%%%%%
%%%%%%%%%%%%%%%%%%%%%%%%%%%%%%%%%%%%%%%%%%%

% размер листа бумаги
\usepackage[paper=a4paper,top=15mm, bottom=15mm,left=35mm,right=10mm,includehead]{geometry}

% всякие разные расстояния
\usepackage{setspace}
\setstretch{1.33}              % Полуторный межстрочный интервал
\setlength{\parindent}{12.5mm}  % Красная строка.

\righthyphenmin=2    % Разрешение переноса двух и более символов
\widowpenalty=10000  % Наказание за вдовствующую строку (одна строка абзаца на этой странице, остальное --- на следующей)
%\clubpenalty=10000  % Наказание за сиротствующую строку (омерзительно висящая одинокая строка в начале страницы)
\tolerance=1000      % Ещё какое-то наказание.

% Нумерация страниц сверху по центру
\usepackage{fancyhdr}
\pagestyle{fancy}
\fancyhead{ } % clear all fields
\fancyfoot{ } % clear all fields
\fancyhead[C]{\thepage}
% Чтобы не прорисовывалась черта!
\renewcommand{\headrulewidth}{0pt}

% Нумерация страниц с надписью "Глава"
\usepackage{etoolbox}
\patchcmd{\chapter}{\thispagestyle{plain}}{\thispagestyle{fancy}}{}{}

% Заголовки по левому краю
% опция identfirst устанавливает отступ в первом абзаце
\usepackage[indentfirst]{titlesec}{\raggedleft}

% В Linux этот пакет для заголовков. Исправляет это следующий непонятный кусок кода:
\makeatletter
\patchcmd{\ttlh@hang}{\parindent\z@}{\parindent\z@\leavevmode}{}{}
\patchcmd{\ttlh@hang}{\noindent}{}{}{}
\makeatother

% Редактирования Глав и названий
\titleformat{\chapter}
{\normalfont\large\bfseries\hyphenpenalty=10000}
{\thechapter }{0.5 em}{}

% Редактирование ненумеруемых глав chapter* (Введение и тп)
\titleformat{name=\chapter,numberless}
{\centering\normalfont\bfseries\large}{}{0.25em}{}

% Убирает чеканутые отступы вверху страницы
\titlespacing{\chapter}{0pt}{-\baselineskip}{\baselineskip}

% Более низкие уровни (подзаголовки)
\titleformat{\section}{\bfseries}{\thesection}{0.5 em}{}
\titleformat{\subsection}{\bfseries}{\thesubsection}{0.5 em}{}

\titlespacing*{\section}{0 pt}{\baselineskip}{\baselineskip}
\titlespacing*{\subsection}{0 pt}{\baselineskip}{\baselineskip}

% Содержание, команды ниже изменяют отступы и рисуют точечки!
\usepackage{titletoc}

\titlecontents{chapter}
[1em] %
{\normalsize}
{\contentslabel{1 em}}
{\hspace{-1 em}}
{\normalsize\titlerule*[10pt]{.}\contentspage}

\titlecontents{section}
[3 em] %
{\normalsize}
{\contentslabel{2 em}}
{\hspace{-2 em}}
{\normalsize\titlerule*[10pt]{.}\contentspage}

\titlecontents{subsection}
[6 em] %
{\normalsize}
{\contentslabel{3 em}}
{\hspace{-3 em}}
{\normalsize\titlerule*[10pt]{.}\contentspage}

% Правильные подписи под таблицей и рисунком
% Документация к пакету на русском языке!
\usepackage[singlelinecheck=false]{caption}
\usepackage{subcaption}

\DeclareCaptionStyle{base}%
[justification=centering,indention=0pt]{}
\DeclareCaptionLabelFormat{gostfigure}{Рисунок #2}
\DeclareCaptionLabelFormat{gostsubfigure}{(#2)}
\DeclareCaptionLabelFormat{gosttable}{Таблица #2}
\DeclareCaptionLabelFormat{gostsubtable}{#2)}

\DeclareCaptionLabelSeparator{gost}{~---~}
\DeclareCaptionLabelSeparator{subgost}{~}

\DeclareCaptionStyle{fig01}%
[margin=0mm,justification=centering,indention=0pt,parindent=0pt]%
{margin={3em,3em}}
\captionsetup*[figure]{position=below,style=fig01,labelsep=gost,labelformat=gostfigure,format=hang}
\captionsetup*[subfigure]{position=below,style=fig01,labelsep=subgost,labelformat=gostsubfigure,format=hang}
\renewcommand\thesubfigure{\asbuk{subfigure}}

\DeclareCaptionStyle{tab01}%
[margin=0mm,justification=raggedright,indention=0pt,parindent=0pt]%
{margin={3em,3em}}
\captionsetup*[table]{position=top,style=tab01,labelsep=gost,labelformat=gosttable,format=hang}
\captionsetup*[subtable]{position=top,style=tab01,labelsep=subgost,labelformat=gostsubtable,format=hang}
\renewcommand\thesubtable{\asbuk{subtable}}

% межстрочный отступ в таблице
\renewcommand{\arraystretch}{1}

% многостраничные таблицы под РОССИЙСКИЙ СТАНДАРТ
% ВНИМАНИЕ! Обязательно после пакета caption!
\usepackage{fr-longtable}

%Более гибкие спсики
\usepackage{enumitem}

% сообщаем окружению о том, что существует такая штука, как нумерация русскими буквами
\makeatletter
\AddEnumerateCounter{\asbuk}{\russian@alph}{щ}
\makeatother

% ГОСТОВСКИЕ СПИСКИ

% Первый тип списков. Большая буква.
\newlist{Enumerate}{enumerate}{1}

\setlist[Enumerate,1]{labelsep=0.5em,leftmargin=1.25em,labelwidth=1.25em,parsep=0em,itemsep=0em,topsep=0ex,before={\parskip=-1em},ref=\arabic{Enumeratei},label=\arabic{Enumeratei}.}
\setlist[Enumerate,2]{leftmargin=1.3em,itemsep=0mm,parsep=0em,topsep=0ex,before={\parskip=-1em},ref=\asbuk{enumii},label=\asbuk{enumii}.}
\setlist[Enumerate,3]{leftmargin=2.6em,itemsep=0mm,parsep=0em,topsep=0ex,before={\parskip=-1em},ref=\arabic{enumiii},label=\arabic{enumiii})}

% Второй тип списков. Маленькая буква.
\setlist[enumerate,1]{parsep=0em,itemsep=0em,topsep=0.75ex,before={\parskip=-1em},ref=\arabic{enumi},label=\arabic{enumi}.}
\setlist[enumerate,2]{leftmargin=1.3em,itemsep=0mm,parsep=0em,topsep=0ex,before={\parskip=-1em},ref=\asbuk{enumii},label=\asbuk{enumii}.}
\setlist[enumerate,3]{leftmargin=2.6em,itemsep=0mm,parsep=0em,topsep=0ex,before={\parskip=-1em},ref=\arabic{enumiii},label=\arabic{enumiii})}

% Третий тип списков. Два уровня.
\newlist{twoenumerate}{enumerate}{2}
\setlist[twoenumerate,1]{itemsep=0mm,parsep=0em,topsep=0.75ex,, before={\parskip=-1em},ref=label=\asbuk{twoenumeratei},label=\asbuk{twoenumeratei})}
\setlist[twoenumerate,2]{leftmargin=1.3em,itemsep=0mm,parsep=0em,topsep=0ex, before={\parskip=-1em},ref=\arabic{twoenumerateii},label=\arabic{twoenumerateii})}

% Четвёртый тип списков. Список с тире.
\setlist[itemize]{label=--,parsep=0em,itemsep=0em,topsep=0ex, before={\parskip=-1em},after={\parskip=-1em}}

% WARNING WARNING WARNIN!
% Если в списке предложения, то должна по госту стоять точка после цифры => команда Enumerate! Если идет перечень маленьких фактов, не обособляемых предложений то после цифры идет скобка ")" => команда enumerate! Если перечень при этом ещё и двууровневый, то twoenumerate.



%%%%%%%%%% Список литературы %%%%%%%%%%

\usepackage[backend=biber, style=gost-numeric, maxbibnames=9, maxcitenames=2, uniquelist=false, babel=other, sorting=nyt]{biblatex}

%\addbibresource{diploma.bib} % this might not work in online TeX compilers
\bibliography{bibliography.bib} % and this should work

% Ещё немного настроек
\DeclareFieldFormat{postnote}{#1} %убирает с. и p.
\renewcommand*{\mkgostheading}[1]{#1} % только лишь убираем курсив с авторов

% Этот кусок кода выносит русские источники на первое место. Костыль описали авторы пакета в руководстве к нему. Подробнее смотри:
% https://github.com/odomanov/biblatex-gost/wiki/Как-сделать%2C-чтобы-русскоязычные-источники-предшествовали-остальным
\DeclareSourcemap{
  \maps[datatype=bibtex]{
    \map{
      \step[fieldsource=langid, match=russian, final]
      \step[fieldset=presort, fieldvalue={a}]
    }
    \map{
      \step[fieldsource=langid, notmatch=russian, final]
      \step[fieldset=presort, fieldvalue={z}]
    }
  }
}

\DefineBibliographyStrings{english}{%
    pages = {P\adddot},
    number = {№},
}

%%Попытка сквозной нумерации сносок
%\usepackage{chngcntr}
%\counterwithout{footnote}{chapter}

%%%%%%%%%% Список литературы %%%%%%%%%%
