%!TEX root = ../main.tex

%%%%%%%%%% Работа с таблицами %%%%%%%%%%
\usepackage{tabularx}           % новые типы колонок
\usepackage{tabulary}           % и ещё новые типы колонок
\usepackage{array,delarray}     % Дополнительная работа с таблицами
\usepackage{longtable}          % Длинные таблицы
\usepackage{multicol}           % несколько колонок
\usepackage{multirow}           % Слияние строк в таблице
\usepackage{float}              % возможность позиционировать объекты в нужном месте
\usepackage{booktabs}           % таблицы как в книгах
\usepackage{adjustbox}          % Сжатие таблиц по ширине
\usepackage{siunitx}
% Заповеди из документации к booktabs:
% 1. Будь проще! Глазам должно быть комфортно
% 2. Не используйте вертикальные линни
% 3. Не используйте двойные линии. Как правило, достаточно трёх горизонтальных линий
% 4. Единицы измерения - в шапку таблицы
% 5. Не сокращайте .1 вместо 0.1
% 6. Повторяющееся значение повторяйте, а не говорите "то же"
% 7. Есть сомнения? Выравнивай по левому краю!

% вычисляемые колонки по tabularx
\newcolumntype{C}{>{\centering\arraybackslash}X}
\newcolumntype{L}{>{\raggedright\arraybackslash}X}
\newcolumntype{Y}{>{\arraybackslash}X}
\newcolumntype{Z}{>{\centering\arraybackslash}X}
